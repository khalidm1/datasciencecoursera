% Options for packages loaded elsewhere
\PassOptionsToPackage{unicode}{hyperref}
\PassOptionsToPackage{hyphens}{url}
%
\documentclass[
]{article}
\usepackage{lmodern}
\usepackage{amssymb,amsmath}
\usepackage{ifxetex,ifluatex}
\ifnum 0\ifxetex 1\fi\ifluatex 1\fi=0 % if pdftex
  \usepackage[T1]{fontenc}
  \usepackage[utf8]{inputenc}
  \usepackage{textcomp} % provide euro and other symbols
\else % if luatex or xetex
  \usepackage{unicode-math}
  \defaultfontfeatures{Scale=MatchLowercase}
  \defaultfontfeatures[\rmfamily]{Ligatures=TeX,Scale=1}
\fi
% Use upquote if available, for straight quotes in verbatim environments
\IfFileExists{upquote.sty}{\usepackage{upquote}}{}
\IfFileExists{microtype.sty}{% use microtype if available
  \usepackage[]{microtype}
  \UseMicrotypeSet[protrusion]{basicmath} % disable protrusion for tt fonts
}{}
\makeatletter
\@ifundefined{KOMAClassName}{% if non-KOMA class
  \IfFileExists{parskip.sty}{%
    \usepackage{parskip}
  }{% else
    \setlength{\parindent}{0pt}
    \setlength{\parskip}{6pt plus 2pt minus 1pt}}
}{% if KOMA class
  \KOMAoptions{parskip=half}}
\makeatother
\usepackage{xcolor}
\IfFileExists{xurl.sty}{\usepackage{xurl}}{} % add URL line breaks if available
\IfFileExists{bookmark.sty}{\usepackage{bookmark}}{\usepackage{hyperref}}
\hypersetup{
  pdftitle={Statistical Inference Course Project Part1},
  pdfauthor={Khalid},
  hidelinks,
  pdfcreator={LaTeX via pandoc}}
\urlstyle{same} % disable monospaced font for URLs
\usepackage[margin=1in]{geometry}
\usepackage{color}
\usepackage{fancyvrb}
\newcommand{\VerbBar}{|}
\newcommand{\VERB}{\Verb[commandchars=\\\{\}]}
\DefineVerbatimEnvironment{Highlighting}{Verbatim}{commandchars=\\\{\}}
% Add ',fontsize=\small' for more characters per line
\usepackage{framed}
\definecolor{shadecolor}{RGB}{248,248,248}
\newenvironment{Shaded}{\begin{snugshade}}{\end{snugshade}}
\newcommand{\AlertTok}[1]{\textcolor[rgb]{0.94,0.16,0.16}{#1}}
\newcommand{\AnnotationTok}[1]{\textcolor[rgb]{0.56,0.35,0.01}{\textbf{\textit{#1}}}}
\newcommand{\AttributeTok}[1]{\textcolor[rgb]{0.77,0.63,0.00}{#1}}
\newcommand{\BaseNTok}[1]{\textcolor[rgb]{0.00,0.00,0.81}{#1}}
\newcommand{\BuiltInTok}[1]{#1}
\newcommand{\CharTok}[1]{\textcolor[rgb]{0.31,0.60,0.02}{#1}}
\newcommand{\CommentTok}[1]{\textcolor[rgb]{0.56,0.35,0.01}{\textit{#1}}}
\newcommand{\CommentVarTok}[1]{\textcolor[rgb]{0.56,0.35,0.01}{\textbf{\textit{#1}}}}
\newcommand{\ConstantTok}[1]{\textcolor[rgb]{0.00,0.00,0.00}{#1}}
\newcommand{\ControlFlowTok}[1]{\textcolor[rgb]{0.13,0.29,0.53}{\textbf{#1}}}
\newcommand{\DataTypeTok}[1]{\textcolor[rgb]{0.13,0.29,0.53}{#1}}
\newcommand{\DecValTok}[1]{\textcolor[rgb]{0.00,0.00,0.81}{#1}}
\newcommand{\DocumentationTok}[1]{\textcolor[rgb]{0.56,0.35,0.01}{\textbf{\textit{#1}}}}
\newcommand{\ErrorTok}[1]{\textcolor[rgb]{0.64,0.00,0.00}{\textbf{#1}}}
\newcommand{\ExtensionTok}[1]{#1}
\newcommand{\FloatTok}[1]{\textcolor[rgb]{0.00,0.00,0.81}{#1}}
\newcommand{\FunctionTok}[1]{\textcolor[rgb]{0.00,0.00,0.00}{#1}}
\newcommand{\ImportTok}[1]{#1}
\newcommand{\InformationTok}[1]{\textcolor[rgb]{0.56,0.35,0.01}{\textbf{\textit{#1}}}}
\newcommand{\KeywordTok}[1]{\textcolor[rgb]{0.13,0.29,0.53}{\textbf{#1}}}
\newcommand{\NormalTok}[1]{#1}
\newcommand{\OperatorTok}[1]{\textcolor[rgb]{0.81,0.36,0.00}{\textbf{#1}}}
\newcommand{\OtherTok}[1]{\textcolor[rgb]{0.56,0.35,0.01}{#1}}
\newcommand{\PreprocessorTok}[1]{\textcolor[rgb]{0.56,0.35,0.01}{\textit{#1}}}
\newcommand{\RegionMarkerTok}[1]{#1}
\newcommand{\SpecialCharTok}[1]{\textcolor[rgb]{0.00,0.00,0.00}{#1}}
\newcommand{\SpecialStringTok}[1]{\textcolor[rgb]{0.31,0.60,0.02}{#1}}
\newcommand{\StringTok}[1]{\textcolor[rgb]{0.31,0.60,0.02}{#1}}
\newcommand{\VariableTok}[1]{\textcolor[rgb]{0.00,0.00,0.00}{#1}}
\newcommand{\VerbatimStringTok}[1]{\textcolor[rgb]{0.31,0.60,0.02}{#1}}
\newcommand{\WarningTok}[1]{\textcolor[rgb]{0.56,0.35,0.01}{\textbf{\textit{#1}}}}
\usepackage{graphicx}
\makeatletter
\def\maxwidth{\ifdim\Gin@nat@width>\linewidth\linewidth\else\Gin@nat@width\fi}
\def\maxheight{\ifdim\Gin@nat@height>\textheight\textheight\else\Gin@nat@height\fi}
\makeatother
% Scale images if necessary, so that they will not overflow the page
% margins by default, and it is still possible to overwrite the defaults
% using explicit options in \includegraphics[width, height, ...]{}
\setkeys{Gin}{width=\maxwidth,height=\maxheight,keepaspectratio}
% Set default figure placement to htbp
\makeatletter
\def\fps@figure{htbp}
\makeatother
\setlength{\emergencystretch}{3em} % prevent overfull lines
\providecommand{\tightlist}{%
  \setlength{\itemsep}{0pt}\setlength{\parskip}{0pt}}
\setcounter{secnumdepth}{-\maxdimen} % remove section numbering

\title{Statistical Inference Course Project Part1}
\author{Khalid}
\date{7/11/2020}

\begin{document}
\maketitle

\hypertarget{part-one}{%
\section{\texorpdfstring{\textbf{Part ONE}}{Part ONE}}\label{part-one}}

\hypertarget{the-project-consists-of-two-parts}{%
\subsubsection{The project consists of two
parts:}\label{the-project-consists-of-two-parts}}

\begin{itemize}
\tightlist
\item
  \textbf{A simulation exercise}.
\item
  Basic inferential data analysis.
\end{itemize}

\hypertarget{investigate-the-exponential-distribution-in-r-and-compare-it-with-the-central-limit-theorem}{%
\subsubsection{investigate the exponential distribution in R and compare
it with the Central Limit
Theorem}\label{investigate-the-exponential-distribution-in-r-and-compare-it-with-the-central-limit-theorem}}

\begin{verbatim}
The exponential distribution can be simulated in R with rexp(n, lambda) where lambda is the rate parameter. 
The mean of exponential distribution is **1/lambda** and the **standard deviation** is also **1/lambda**. 
Set **lambda = 0.2** for all of the simulations. You will investigate the distribution of averages of **40 exponentials**. 
Note that you will need to do a **thousand simulations**.
\end{verbatim}

\hypertarget{illustrate-via-simulation-and-associated-explanatory-text-the-properties-of-the-distribution-of-the-mean-of-40-exponentials.-you-should}{%
\subsubsection{Illustrate via simulation and associated explanatory text
the properties of the distribution of the mean of 40 exponentials. You
should}\label{illustrate-via-simulation-and-associated-explanatory-text-the-properties-of-the-distribution-of-the-mean-of-40-exponentials.-you-should}}

\begin{verbatim}
- Show the sample mean and compare it to the theoretical mean of the distribution.
- Show how variable the sample is (via variance) and compare it to the theoretical variance of the distribution.
- Show that the distribution is approximately normal.
\end{verbatim}

\hypertarget{install-needed-packages}{%
\subsubsection{install needed packages}\label{install-needed-packages}}

\begin{Shaded}
\begin{Highlighting}[]
\KeywordTok{library}\NormalTok{(cowplot)}
\end{Highlighting}
\end{Shaded}

\begin{verbatim}
## 
## ********************************************************
\end{verbatim}

\begin{verbatim}
## Note: As of version 1.0.0, cowplot does not change the
\end{verbatim}

\begin{verbatim}
##   default ggplot2 theme anymore. To recover the previous
\end{verbatim}

\begin{verbatim}
##   behavior, execute:
##   theme_set(theme_cowplot())
\end{verbatim}

\begin{verbatim}
## ********************************************************
\end{verbatim}

\begin{Shaded}
\begin{Highlighting}[]
\KeywordTok{library}\NormalTok{(ggplot2)}
\end{Highlighting}
\end{Shaded}

\hypertarget{assigning-values}{%
\subsubsection{assigning values}\label{assigning-values}}

\begin{Shaded}
\begin{Highlighting}[]
\NormalTok{lambda \textless{}{-}}\StringTok{ }\FloatTok{0.2}
\NormalTok{sample\_num \textless{}{-}}\StringTok{ }\DecValTok{40}
\NormalTok{sim\_num \textless{}{-}}\StringTok{ }\DecValTok{1000}
\end{Highlighting}
\end{Shaded}

\hypertarget{calculate-the-40-samples-1000-times}{%
\subsubsection{calculate the 40 samples 1000
times}\label{calculate-the-40-samples-1000-times}}

\begin{Shaded}
\begin{Highlighting}[]
\KeywordTok{set.seed}\NormalTok{(}\DecValTok{2}\NormalTok{) }\CommentTok{\#for reproducibility purposes}
\NormalTok{mean\_sim\_num \textless{}{-}}\StringTok{ }\KeywordTok{replicate}\NormalTok{(}\DecValTok{1000}\NormalTok{, }\KeywordTok{mean}\NormalTok{(}\KeywordTok{rexp}\NormalTok{(sample\_num, }\DataTypeTok{rate =}\NormalTok{ lambda)))}
\end{Highlighting}
\end{Shaded}

\hypertarget{show-the-sample-mean-and-compare-it-to-the-theoretical-mean-of-the-distribution.}{%
\subsubsection{Show the sample mean and compare it to the theoretical
mean of the
distribution.}\label{show-the-sample-mean-and-compare-it-to-the-theoretical-mean-of-the-distribution.}}

\hypertarget{the-mean-of-exponential-distribution-is-1lambda}{%
\subsubsection{The mean of exponential distribution is
1/lambda}\label{the-mean-of-exponential-distribution-is-1lambda}}

\begin{Shaded}
\begin{Highlighting}[]
\DecValTok{1}\OperatorTok{/}\NormalTok{lambda}
\end{Highlighting}
\end{Shaded}

\begin{verbatim}
## [1] 5
\end{verbatim}

\begin{Shaded}
\begin{Highlighting}[]
\CommentTok{\# The mean othe 40 samples 1000 times}
\KeywordTok{mean}\NormalTok{(mean\_sim\_num)}
\end{Highlighting}
\end{Shaded}

\begin{verbatim}
## [1] 5.016356
\end{verbatim}

\hypertarget{visualization}{%
\subsubsection{visualization}\label{visualization}}

\begin{Shaded}
\begin{Highlighting}[]
\KeywordTok{hist}\NormalTok{(mean\_sim\_num, }\DataTypeTok{xlab =} \StringTok{"mean"}\NormalTok{, }\DataTypeTok{main =} \StringTok{"Exponential Function Simulations"}\NormalTok{)}
\end{Highlighting}
\end{Shaded}

\includegraphics{Statistical-Inference-Course-Project-Part1_files/figure-latex/unnamed-chunk-6-1.pdf}

\begin{Shaded}
\begin{Highlighting}[]
\KeywordTok{hist}\NormalTok{(mean\_sim\_num, }\DataTypeTok{xlab =} \StringTok{"mean"}\NormalTok{, }\DataTypeTok{main =} \StringTok{"Exponential Function Simulations"}\NormalTok{)}
\KeywordTok{abline}\NormalTok{(}\DataTypeTok{v =}\NormalTok{ mean\_sim\_num, }\DataTypeTok{col =} \StringTok{"red"}\NormalTok{)}
\end{Highlighting}
\end{Shaded}

\includegraphics{Statistical-Inference-Course-Project-Part1_files/figure-latex/unnamed-chunk-7-1.pdf}

\begin{Shaded}
\begin{Highlighting}[]
\KeywordTok{hist}\NormalTok{(mean\_sim\_num, }\DataTypeTok{xlab =} \StringTok{"mean"}\NormalTok{, }\DataTypeTok{main =} \StringTok{"Exponential Function Simulations"}\NormalTok{)}
\KeywordTok{abline}\NormalTok{(}\DataTypeTok{v =}\NormalTok{ mean\_sim\_num, }\DataTypeTok{col =} \StringTok{"pink"}\NormalTok{)}
\end{Highlighting}
\end{Shaded}

\includegraphics{Statistical-Inference-Course-Project-Part1_files/figure-latex/unnamed-chunk-8-1.pdf}

\begin{Shaded}
\begin{Highlighting}[]
\KeywordTok{ggplot}\NormalTok{(}\DataTypeTok{data =} \KeywordTok{as.data.frame}\NormalTok{(mean\_sim\_num), }\KeywordTok{aes}\NormalTok{(}\DataTypeTok{x =}\NormalTok{ mean\_sim\_num)) }\OperatorTok{+}
\StringTok{        }\KeywordTok{geom\_histogram}\NormalTok{(}\DataTypeTok{binwidth =} \FloatTok{0.1}\NormalTok{, }\KeywordTok{aes}\NormalTok{(}\DataTypeTok{y =}\NormalTok{ ..density..), }\DataTypeTok{alpha =} \FloatTok{0.3}\NormalTok{) }\OperatorTok{+}
\StringTok{        }\KeywordTok{geom\_density}\NormalTok{(}\DataTypeTok{color =} \StringTok{\textquotesingle{}black\textquotesingle{}}\NormalTok{) }\OperatorTok{+}\StringTok{ }\CommentTok{\#density curve of the sample distribution}
\StringTok{        }\KeywordTok{geom\_vline}\NormalTok{(}\DataTypeTok{xintercept =} \KeywordTok{mean}\NormalTok{(mean\_sim\_num), }\DataTypeTok{color =} \StringTok{\textquotesingle{}black\textquotesingle{}}\NormalTok{) }\OperatorTok{+}
\StringTok{        }\KeywordTok{stat\_function}\NormalTok{(}\DataTypeTok{fun =}\NormalTok{ dnorm, }\DataTypeTok{args =} \KeywordTok{list}\NormalTok{(}\DataTypeTok{mean =} \DecValTok{1}\OperatorTok{/}\NormalTok{lambda, }\DataTypeTok{sd =} \DecValTok{1}\OperatorTok{/}\NormalTok{lambda}\OperatorTok{/}\KeywordTok{sqrt}\NormalTok{(sample\_num)), }\DataTypeTok{color =} \StringTok{\textquotesingle{}red\textquotesingle{}}\NormalTok{) }\OperatorTok{+}
\StringTok{        }\KeywordTok{geom\_vline}\NormalTok{(}\DataTypeTok{xintercept =} \DecValTok{1}\OperatorTok{/}\NormalTok{lambda, }\DataTypeTok{color =} \StringTok{\textquotesingle{}red\textquotesingle{}}\NormalTok{) }\OperatorTok{+}
\StringTok{        }\KeywordTok{xlab}\NormalTok{(}\StringTok{\textquotesingle{}Sample mean\textquotesingle{}}\NormalTok{) }\OperatorTok{+}
\StringTok{        }\KeywordTok{ylab}\NormalTok{(}\StringTok{\textquotesingle{}Density\textquotesingle{}}\NormalTok{) }\OperatorTok{+}
\StringTok{        }\KeywordTok{ggtitle}\NormalTok{(}\StringTok{\textquotesingle{}Histogram of 1000 simulations for mean of 40 samples\textquotesingle{}}\NormalTok{)}
\end{Highlighting}
\end{Shaded}

\includegraphics{Statistical-Inference-Course-Project-Part1_files/figure-latex/unnamed-chunk-9-1.pdf}

\hypertarget{show-how-variable-the-sample-is-via-variance-and-compare-it-to-the-theoretical-variance-of-the-distribution.}{%
\subsubsection{Show how variable the sample is (via variance) and
compare it to the theoretical variance of the
distribution.}\label{show-how-variable-the-sample-is-via-variance-and-compare-it-to-the-theoretical-variance-of-the-distribution.}}

\hypertarget{standard-deviation-of-distribution}{%
\paragraph{standard deviation of
distribution}\label{standard-deviation-of-distribution}}

\begin{Shaded}
\begin{Highlighting}[]
\NormalTok{sd\_dv\_dist \textless{}{-}}\StringTok{ }\KeywordTok{sd}\NormalTok{(mean\_sim\_num)}
\NormalTok{sd\_dv\_dist}
\end{Highlighting}
\end{Shaded}

\begin{verbatim}
## [1] 0.818004
\end{verbatim}

\hypertarget{standard-deviation-from-analytical-expression}{%
\paragraph{standard deviation from analytical
expression}\label{standard-deviation-from-analytical-expression}}

\begin{Shaded}
\begin{Highlighting}[]
\NormalTok{sd\_dv\_theory \textless{}{-}}\StringTok{ }\NormalTok{(}\DecValTok{1}  \OperatorTok{/}\StringTok{ }\NormalTok{lambda) }\OperatorTok{/}\StringTok{ }\KeywordTok{sqrt}\NormalTok{(sample\_num)}
\NormalTok{sd\_dv\_theory}
\end{Highlighting}
\end{Shaded}

\begin{verbatim}
## [1] 0.7905694
\end{verbatim}

\hypertarget{variance-of-distribution}{%
\paragraph{variance of distribution}\label{variance-of-distribution}}

\begin{Shaded}
\begin{Highlighting}[]
\NormalTok{variance\_dist \textless{}{-}}\StringTok{ }\NormalTok{sd\_dv\_dist}\OperatorTok{\^{}}\DecValTok{2}
\NormalTok{variance\_dist}
\end{Highlighting}
\end{Shaded}

\begin{verbatim}
## [1] 0.6691305
\end{verbatim}

\hypertarget{variance-from-analytical-expression}{%
\paragraph{variance from analytical
expression}\label{variance-from-analytical-expression}}

\begin{Shaded}
\begin{Highlighting}[]
\NormalTok{variance\_theory \textless{}{-}}\StringTok{ }\NormalTok{((}\DecValTok{1} \OperatorTok{/}\StringTok{ }\NormalTok{lambda)}\OperatorTok{*}\NormalTok{(}\DecValTok{1} \OperatorTok{/}\StringTok{ }\KeywordTok{sqrt}\NormalTok{(sample\_num)))}\OperatorTok{\^{}}\DecValTok{2}
\NormalTok{variance\_theory}
\end{Highlighting}
\end{Shaded}

\begin{verbatim}
## [1] 0.625
\end{verbatim}

\hypertarget{show-that-the-distribution-is-approximately-normal.}{%
\paragraph{Show that the distribution is approximately
normal.}\label{show-that-the-distribution-is-approximately-normal.}}

\begin{Shaded}
\begin{Highlighting}[]
\NormalTok{xfit \textless{}{-}}\StringTok{ }\KeywordTok{seq}\NormalTok{(}\KeywordTok{min}\NormalTok{(mean\_sim\_num), }\KeywordTok{max}\NormalTok{(mean\_sim\_num), }\DataTypeTok{length=}\DecValTok{100}\NormalTok{)}
\NormalTok{yfit \textless{}{-}}\StringTok{ }\KeywordTok{dnorm}\NormalTok{(xfit, }\DataTypeTok{mean=}\DecValTok{1}\OperatorTok{/}\NormalTok{lambda, }\DataTypeTok{sd=}\NormalTok{(}\DecValTok{1}\OperatorTok{/}\NormalTok{lambda}\OperatorTok{/}\KeywordTok{sqrt}\NormalTok{(sample\_num)))}
\end{Highlighting}
\end{Shaded}

\begin{Shaded}
\begin{Highlighting}[]
\KeywordTok{hist}\NormalTok{(mean\_sim\_num,}\DataTypeTok{breaks=}\NormalTok{sample\_num,}\DataTypeTok{prob=}\NormalTok{T,}\DataTypeTok{col=}\StringTok{"blue"}\NormalTok{,}\DataTypeTok{xlab =} \StringTok{"means"}\NormalTok{,}\DataTypeTok{main=}\StringTok{"Density of means"}\NormalTok{,}\DataTypeTok{ylab=}\StringTok{"density"}\NormalTok{)}
\end{Highlighting}
\end{Shaded}

\includegraphics{Statistical-Inference-Course-Project-Part1_files/figure-latex/unnamed-chunk-15-1.pdf}

\begin{Shaded}
\begin{Highlighting}[]
\KeywordTok{hist}\NormalTok{(mean\_sim\_num,}\DataTypeTok{breaks=}\NormalTok{sample\_num,}\DataTypeTok{prob=}\NormalTok{T,}\DataTypeTok{col=}\StringTok{"blue"}\NormalTok{,}\DataTypeTok{xlab =} \StringTok{"means"}\NormalTok{,}\DataTypeTok{main=}\StringTok{"Density of means"}\NormalTok{,}\DataTypeTok{ylab=}\StringTok{"density"}\NormalTok{)}
\KeywordTok{lines}\NormalTok{(xfit, yfit, }\DataTypeTok{pch=}\DecValTok{22}\NormalTok{, }\DataTypeTok{col=}\StringTok{"red"}\NormalTok{, }\DataTypeTok{lty=}\DecValTok{5}\NormalTok{)}
\end{Highlighting}
\end{Shaded}

\includegraphics{Statistical-Inference-Course-Project-Part1_files/figure-latex/unnamed-chunk-16-1.pdf}

\hypertarget{compare-the-distribution-of-averages-of-40-exponentials-to-a-normal-distribution}{%
\paragraph{compare the distribution of averages of 40 exponentials to a
normal
distribution}\label{compare-the-distribution-of-averages-of-40-exponentials-to-a-normal-distribution}}

\begin{Shaded}
\begin{Highlighting}[]
\KeywordTok{qqnorm}\NormalTok{(mean\_sim\_num)}
\KeywordTok{qqline}\NormalTok{(mean\_sim\_num, }\DataTypeTok{col =} \DecValTok{2}\NormalTok{)}
\end{Highlighting}
\end{Shaded}

\includegraphics{Statistical-Inference-Course-Project-Part1_files/figure-latex/unnamed-chunk-17-1.pdf}

\end{document}
